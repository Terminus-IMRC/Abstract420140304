\documentclass[a4paper,landscape,25pt]{foils}
\usepackage{graphicx}

\title{5次魔方陣全解出力プログラムの \\ 並列化と性能評価}
\author{4年次 \\ 杉崎 行優}

\begin{document}
\maketitle

\foilhead[-.5in]{5次魔方陣}
\begin{itemize}
\item 定義
\begin{itemize}
\item 5 $\times$ 5 のマス目
\item 1 $\sim$ 25 の数が1つずつ入る
\item 縦の5列 $\cdot$ 横の5列 $\cdot$ 斜めの2列の合計が全て等しい(式\ref{eqn:l})
\begin{boldequation} \label{eqn:l}
L=\frac{1}{5}\sum_{i=1}^{25}i=\frac{5\times(25+1)}{2}=65
\end{boldequation}
\end{itemize}
\item 全解は2億7530万5224通り存在
\end{itemize}

\foilhead{5次魔方陣の例}
\begin{figure}[htb]
\centering
\includegraphics[height=0.7\textheight]{tmp}
%%\caption{5次魔方陣の例}
\end{figure}

\foilhead{アルゴリズム}
\begin{itemize}
\item 全ての数字を総当たりで入れるのが基本
\item 列の和より、5マス中の4マスが埋まれば残りの1マスは自動的に求められる(枝刈り)
\begin{itemize}
\item 総当りするマスを14マスに減らす
\item 完全総当たりより約$4\times10^7$倍高速化
\end{itemize}
\end{itemize}

\foilhead{並列化$(1)$}
\begin{itemize}
\item 並列方式 $\cdots$ マスタ$\cdot$ワーカー型並列
\begin{itemize}
\item 1コアをマスタ(司令塔)とする
\item その他のコアをワーカーとし、マスタの指示に従わせる
\end{itemize}
\end{itemize}

\foilhead{並列化$(2)$}
\begin{itemize}
\item マスタが$N$番目$(0 \ge N \ge 14)$のマスまで総当たりし、ワーカーがそれを受け取り、$N+1$番目のマスまで総当たり
\begin{itemize}
\item $N$の値が小さいとワーカーの粒度(仕事のバラつき)が大きくなり、ワーカーの計算時間がバラつく
\item $N$の値が大きいと粒度が小さくなりワーカーの計算時間が均等になるが、通信のコストが増大
\end{itemize}
\end{itemize}


\end{document}
