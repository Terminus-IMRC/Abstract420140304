\documentclass[a4paper,landscape,25pt]{foils}

\title{5次魔方陣全解出力プログラムの \\ 並列化と性能評価}
\author{4年次 \\ 杉崎 行優}

\begin{document}
\maketitle

\foilhead[-.5in]{5次魔方陣}
\begin{itemize}
\item 定義
\begin{itemize}
\item 5 $\times$ 5 のマス目
\item 1 $\sim$ 25 の数を1つずつ入れる
\item 縦の5列 $\cdot$ 横の5列 $\cdot$ 斜めの2列の合計が全て等しい(式\ref{eqn:l})
\begin{boldequation} \label{eqn:l}
L=\frac{1}{5}\sum_{i=1}^{25}i=\frac{5\times(25+1)}{2}=65
\end{boldequation}
\end{itemize}
\item 2億7530万5224通りある
\end{itemize}

\foilhead{5次魔方陣の例}

\foilhead{アルゴリズム}
\begin{itemize}
\item 全ての数字を総当たりで入れるのが基本
\item 列の和より、5マス中の4マスが埋まれば残りの1マスは自動的に求められる(枝刈り)
\begin{itemize}
\item 総当りするマスを減らすことが可能
\item 完全総当たりより
\end{itemize}
\end{itemize}

\end{document}
